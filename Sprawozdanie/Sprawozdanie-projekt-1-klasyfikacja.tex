\documentclass{classrep}
\usepackage[utf8]{inputenc}
\usepackage{color}

\studycycle{Informatyka, studia STACJONARNE, I st.}
\coursesemester{VI}

\coursename{Komputerowe systemy rozpoznawania}
\courseyear{2020/2021}

\courseteacher{prof. dr hab. inż. Adam Niewiadomski}
\coursegroup{poniedziałek, 12:00}

\author{
  \studentinfo{Julia Szymańska}{224441} \and
  \studentinfo{Przemysław Zdrzalik}{224466} }

\title{Projekt 1. Klasyfikacja dokumentów tekstowych}
\usepackage{multirow}
\begin{document}
\maketitle


\section{Cel projektu}
Celem projektu jest stworzenie aplikacji klasyfikującej zadany zbiór danych tekstowych metodą K najbliższych sąsiadów (k-NN). Aplikacja ma za zadanie dokonać ekstrakcji cech na zbiorach tekstów ze strony \\\textit{http://archive.ics.uci.edu/ml/datasets/Reuters-21578+Text+Categorization+Collection} oraz następnie dokonać ich klasyfikacji.\\


\section{Klasyfikacja nadzorowana metodą $k$-NN}
\color{red} POPRAWIC, dac linka do zrodel
\color{black}
Metoda K najbliższych sąsiadów, w skrócie metoda $k$-NN, jest to algorytm stosowany do klasyfikacji, który nie wymaga etapu uczenia. 
Polega na zaklasyfikowaniu rozpatrywanego elementu do grupy ze zbioru uczącego, gdzie spośród k najbliższych rozpatrywanemu elementowi sąsiadów najwięcej z nich należy do tej grupy. Klasyfikator przyjmuje cztery parametry wejściowe takie jak: warotść k - ilość rozpatrywanych sąsiadów, proporcje podziału zbiorów na zbior uczący i zbiór testowy, zbiór cech, a także metrykę i/lub miarę prawdopodobieństwa. Wynikiem klasyfikacji jest zaklasyfikowanie elementu do jednego ze zbiorów uczących. 


\subsection{Ekstrakcja cech, wektory cech}
Na zbiorach danych tekstowych należy dokonać ekstrakcji cech, które będą wartościami rzeczywistymi oraz tekstowymi. Dane cechy będą reprezentowały tekst w postaci wektora cech podczas procesu klasyfikacji. Przed dokonaniem ekstrakcji cech, z tekstów usuwane są słowa znajdujące się na stop liście. Teksty ze zbioru danych tekstowych posiadają strukturę: \begin{equation}
  \begin{array}{l}
  <TEXT> \\
\;\;\;\; <TITLE/>\\
\;\;\;\; <AUTHOR/>\\
\;\;\;\; <DATELINE/>\\
 \;\;\;\;<BODY/> \\
</TEXT>
  \end{array}
\end{equation}\\
\begin{enumerate}
  \item Liczba słów - cecha ta oznacza liczbę słów które składają się na pobrany tekst. Cecha ta będzie charakteryzowała długość dokumentu w postaci liczby całkowitej \begin{equation}  c_1 = len \end{equation} gdzie len - liczba słów w tekście po dokonaniu redukcji.\\
  
\item Druga najczęściej występująca waluta - wybieramy drugą najczęściej występującą walutę z tekstu, ponieważ uważamy, że pierwszą najczęściej występującą walutą będzie dolar ze względu na jego powszechne zastosowanie. Do pobierania nazw walut wykorzystujemy dołączony plik gdzie znajduje się 27 różnych walut wraz z kodami jakimi reprezentowane są w pobieranych tekstach. Przykładem jest kod dla waluty Dolaru Amerykańskiego - DLR. Cechę traktujemy jako cechę tekstową.Wartość będzie oznaczana poprzez symbol  c\textsubscript{2}. \\
 
 \item Data z tagu  \textless Dateline\textgreater\ - Każdy tekst w swoim body posiada tag \textless Dateline\textgreater , w którym znajduje się miasto oraz data podana w postaci miesiąca i dnia. Data będzie konwertowana na wartość liczbową, gdzie liczbą tą będzie numer podanego dnia w ciągu roku, licząc rok tak jakby rok był rokiem przestępnym, przykładowo data 1 marca będzie reprezentowana poprzez wartość 61. Cechę traktujemy jako cechę w postaci liczby całkowitej. Wartość będzie oznaczana poprzez symbol  c\textsubscript{3}.    \\
  \item Lokacja z tagu \textless Dateline\textgreater - jak wyżej. Lokację traktujemy jako cechę tekstową. Wartość będzie oznaczana poprzez symbol  c\textsubscript{4}. \\
  \item Tytuł z tagu \textless Title\textgreater - Każdy tekst w swoim body posiada tag \textless Title\textgreater. Tytuł traktujemy jako cechę tekstową. Wartość będzie oznaczana poprzez symbol  c\textsubscript{5}.\\
  \item Autor z tagu \textless Author\textgreater - Większość tekstów w swoim body posiada tag \textless Author\textgreater. Autora traktujemy jako cechę tekstową. Wartość będzie oznaczana poprzez symbol  c\textsubscript{6}.\\
  \item Najczęściej występująca nazwa kraju - wybieramy najczęściej występującą w analizowanym tekście nazwę kraju. Nazwy krajów pobieramy z dołączonego pliku all-places-strings.lc, przykładowo krajem występującym w pliku jest 'albania'.Nazwę kraju traktujemy jako cechę tekstową.Wartość będzie oznaczana poprzez symbol  c\textsubscript{7}.\\
  \item Zbiór występujących słów kluczowych. Za słowa kluczowe przyjmujemy słowa znajdujące się w dołączonych plikach o rozszerzeniach .lc.txt. Cechę traktujemy jako cechę tekstową.  \begin{equation}  c_8 : c_8 \in N \cap t \end{equation} gdzie N - zbiór wszystkich słów kluczowych, t - zbiór słów należących do tekstu\\
  \item Ilość wystąpień słów kluczowych - traktujemy jako cechę w postaci liczby całkowitej.\begin{equation}  c_9 = | c_8 | \end{equation} gdzie c\textsubscript{8} - zbiór występujących słów kluczowych\\
  \item Nasycenie tekstu ilością słów kluczowych - traktujemy jako cechę w postaci liczby zmienno przecinkowej.  \begin{equation} c_{10} = c_9 / c_1 \end{equation}  gdzie c\textsubscript{9} - ilość wystąpień słów kluczowych w tekscie, c\textsubscript{1} - liczba słów w tekście\\
  \item Najczęściej występujące słowo kluczowe - wybieramy najczęściej występujące w analizowanym tekście słowo kluczowe. Cechę traktujemy jako cechę tekstową. Wartość będzie oznaczana poprzez symbol  c\textsubscript{11}.\\
  \item Liczba unikatowych słów - zliczamy liczbę unikatowych słów, to znaczy występujących dokładnie raz w analizowanym tekście. Cechę traktujemy jako cechę w postaci liczby całkowitej. Wartość będzie oznaczana poprzez symbol  c\textsubscript{12}.\\
\end{enumerate}

\ \\ \\
Wektor cech będzie reprezentowany w postaci: 

\begin{equation} w = [c_1, c_2, c_3, c_4, c_5, c_6, c_7, c_8, c_9, c_{10}, c_{11}, c_{12}] \end{equation}



\subsection{Miary jakości klasyfikacji} 
W celu określenia jakości wykonanej klasyfikacji korzystamy z czterech miar jakości klasyfikacji. Aby obliczyć każdą z miar tworzymy tablicę pomyłek, inaczej macierz błędu \cite{tablica}. Tablica składa się z dwóch wierszy i dwóch kolumn, gdzie wiersze to klasy predykowane, a kolumny to klasy rzeczywiste. Dane oznaczone jako dane pozytywne i negatywne poddawane są klasyfikacji, która przypisuje im predykowaną klasę pozytywną bądź snegatywną.\\

\begin{table}[h!]

\begin{tabular}{l|l|c|c|c}
\multicolumn{2}{c}{}&\multicolumn{2}{c}{Klasa rzeczywista}&\\
\cline{3-4}
\multicolumn{2}{c|}{}&Pozytywna&Negatywna&\multicolumn{1}{c}{}\\
\cline{2-4}
\multirow{2}{*}{Klasa predykowana}& Pozytywna& prawdziwie
pozytywna (TP) & fałszywie
pozytywna (FP)\\
\cline{2-4}
& Negatywna & fałszywie
negatywna (FN) & prawdziwie
negatywna (TN) \\
\cline{2-4}
\end{tabular}
 \caption{Wzór tablicy pomyłek.}
\end{table}

\section{Klasyfikacja z użyciem metryk i miar podobieństwa tekstów}
Wzory, znaczenia i opisy symboli zastosowanych metryk z
przykładami. Wzory, opisy i znaczenia miar
podobieństwa tekstów zastosowanych w obliczaniu metryk dla wektorów cech z
przykładami dla każdej miary \cite{niewiadomski08}.  Oznaczenia jednolite w obrębie całego sprawozdania.  Wstępne wyniki miary Accuracy dla próbnych klasyfikacji na ograniczonym zbiorze tekstów (podać parametry i kryteria
wyboru wg punktów 3.-8. z opisu Projektu 1.). \\ 
\noindent {\bf Sekcja uzupełniona jako efekt zadania Tydzień 04 wg Harmonogramu Zajęć
na WIKAMP KSR.}

\section{Budowa aplikacji}
\subsection{Diagramy UML}
Diagramy UML i zwięzłe opisy: idei aplikacji, modułu ekstrakcji i modułu
klasyfikatora.\\
\noindent {\bf Sekcja uzupełniona jako efekt zadania Tydzień 03 wg Harmonogramu Zajęć
na WIKAMP KSR.}

\subsection{Prezentacja wyników, interfejs użytkownika} 
Krótki ilustrowany opis jak użytkownik może korzystać z aplikacji, w~szczególności wprowadzać parametry klasyfikacji i odczytywać wyniki. Wersja JRE i inne wymogi
niezbędne do uruchomienia aplikacji przez użytkownika na własnym komputerze. \\
\noindent {\bf Sekcja uzupełniona jako efekt zadania Tydzień 04 wg Harmonogramu Zajęć
na WIKAMP KSR.}

\section{Wyniki klasyfikacji dla różnych parametrów wejściowych}
Wyniki kolejnych eksperymentów wg punktów 2.-8. opisu projektu 1.  Wykresy i tabele
obowiązkowe, dokładnie opisane w ,,captions'' (tytułach), konieczny opis osi i
jednostek wykresów oraz kolumn i wierszy tabel.\\ 

{**Ewentualne wyniki realizacji punktu 9. opisu Projektu 1., czyli,,na ocenę 5.0'' i ich porównanie do wyników z
części obowiązkowej**.}\\

\noindent {\bf Sekcja uzupełniona jako efekt zadania Tydzień 05 wg Harmonogramu Zajęć
na WIKAMP KSR.}


\section{Dyskusja, wnioski}
Dokładne interpretacje uzyskanych wyników w zależności od parametrów klasyfikacji
opisanych w punktach 3.-8 opisu Projektu 1. 
Szczególnie istotne są wnioski o charakterze uniwersalnym, istotne dla podobnych zadań. 
Omówić i wyjaśnić napotkane problemy (jeśli były). Każdy wniosek/problem powinien mieć poparcie
w przeprowadzonych eksperymentach (odwołania do konkretnych wyników: wykresów,
tabel). \\
\underline{Dla końcowej oceny jest to najważniejsza sekcja} sprawozdania, gdyż prezentuje poziom
zrozumienia rozwiązywanego problemu.\\

** Możliwości kontynuacji prac w obszarze systemów rozpoznawania, zwłaszcza w kontekście pracy inżynierskiej,
magisterskiej, naukowej, itp. **\\

\noindent {\bf Sekcja uzupełniona jako efekt zadania Tydzień 06 wg Harmonogramu Zajęć
na WIKAMP KSR.}


\section{Braki w realizacji projektu 1.}
Wymienić wg opisu Projektu 1. wszystkie niezrealizowane obowiązkowe elementy projektu, ewentualnie
podać merytoryczne (ale nie czasowe) przyczyny tych braków. 


\begin{thebibliography}{0}
\bibitem{dane} R. Tadeusiewicz: Rozpoznawanie obrazów, PWN, Warszawa, 1991.  
\bibitem{niewiadomski08} A. Niewiadomski, Methods for the Linguistic Summarization of Data: Applications of Fuzzy Sets and Their Extensions, Akademicka Oficyna Wydawnicza EXIT, Warszawa, 2008.

\bibitem{tablica} Tablica pomyłek: \\\textit{} 

\end{thebibliography}

Literatura zawiera wyłącznie źródła recenzowane i/lub o potwierdzonej wiarygodności,
możliwe do weryfikacji i cytowane w sprawozdaniu. 
\end{document}



